\documentclass[10pt,svgnames,fragile]{beamer}
\usepackage[frenchb]{babel}
\usepackage[utf8x]{inputenc}
\usepackage{lmodern}
\usepackage[T1]{fontenc}
\usepackage{graphicx}
\usepackage{etex}
\usepackage{xcolor}
\usepackage[normalem]{ulem}
\usepackage{textcomp}
\usepackage{pdflscape}
\usepackage{marvosym}
\usepackage{wasysym}
\usepackage{amssymb}
\usepackage{amsmath}
\usepackage{amsthm}
\usepackage{qtree}
\usepackage{bussproofs}
\usepackage{proof}
\usepackage{fitch}
\usepackage{cancel}
\usepackage{url}
\usepackage{smfthm}
%% Le code qui suit permet le rappel du plan à chaque changement de section :
\AtBeginSection[]{\begin{frame}<beamer>\frametitle{}\tableofcontents[currentsection,hideothersubsections]\end{frame}}

\titlegraphic{\includegraphics[height=1.5cm]{ssd}}
\institute[Université de Montpellier]{Master 1 SSD}
\usetheme{CambridgeUS}
\usepackage{beamer_udl_theme}
\setbeamertemplate{navigation symbols}{}
\author{} % Auteurs supprimés de la zone en bas à gauche
\date{}
\title{ATVM \\ 
\vspace{0.5cm}
\small Wahel El Mazzouji, Riwa Hachem, Reda Lucien Duigou, Lamia Oulebsir}
\begin{document}

\maketitle

\section{Introduction}
\label{sec:introduction}
\begin{frame}[label={sec:orge9abdcb}]{Introduction}
\begin{itemize}
    \item Projet centré sur l’analyse et la visualisation des trajets Velomagg à Montpellier.
    \item Objectifs principaux :
    \begin{itemize}
        \item Traiter et analyser des données pour mieux comprendre les trajets.
        \item Créer des visualisations interactives et dynamiques.
        \item Prédictions
    \end{itemize}
\end{itemize}
\end{frame}
%%%%%%%%%%%%%%%%%%%%%%%%%%%%%%%%%%% PEUT ETRE PARLER DU SITE AVANT
\section{Données utilisées}
\label{sec:data}
\begin{frame}[label={sec:dataframe}]{Données utilisées}
\begin{block}{Sources des données}
\begin{itemize}
    \item Données des trajets des vélos VéloMagg : Informations détaillées sur les trajets effectués avec les vélos en libre-service de Montpellier Méditerranée Métropole.
    \item Données de comptage : Statistiques issues des compteurs de vélos et piétons installés dans la ville, permettant d'analyser les flux de trafic.
    \item Données OpenStreetMap : Cartes et informations géographiques pour la localisation et la représentation des stations et des trajets.
    \item Données ouvertes d'Open Data Montpellier.
\end{itemize}
\end{block}
\end{frame}

\begin{frame}[label={sec:org88fd8be}]{Prétraitement des données}
\begin{columns}
    \begin{column}{0.5\textwidth}
        \begin{block}{Étapes clés}
        \begin{itemize}
            \item Modification des encodages pour corriger les caractères spéciaux et filtrage des stations non pertinentes.
            \item Attribution des coordonnées GPS aux stations restantes.
            \item Sauvegarde des résultats dans un fichier JSON.
        \end{itemize}
        \end{block}
    \end{column}
    \begin{column}{0.5\textwidth}
        \begin{center}
            \resizebox{\textwidth}{!}{\includegraphics{coord_stations.png}}  % Image dans la deuxième colonne
        \end{center}
    \end{column}
\end{columns}
\end{frame}

\section{Visualisations interactives}
\label{sec:visualizations}
\begin{frame}[label={sec:visualizationsday}]{Cartes des trajets d’une journée}
\begin{block}{Présentation}
\begin{itemize}
    \item Carte affichant tous les trajets effectués en une journée.
    \item Date : 13/07/2024.
    \item Nombre de trajets affichés : 60 pour 580 au total.
\end{itemize}
\end{block}

\begin{columns}
    \begin{column}{0.6\textwidth}
        \includegraphics[width=\textwidth]{map_1.png} 
    \end{column}

    % Deuxième colonne : légende
    \begin{column}{0.4\textwidth}
        \begin{block}{Légende}
            \begin{itemize}
                \item \textcolor{green}{\textbullet} Moins de 1 km : Trajets courts, en vert.
                \item \textcolor{blue}{\textbullet} Entre 1 et 2 km : Trajets moyens, en bleu.
                \item \textcolor{orange}{\textbullet} Entre 2 et 3 km : Trajets longs, en orange.
                \item \textcolor{red}{\textbullet} Plus de 5 km : Trajets très longs, en rouge.
            \end{itemize}
        \end{block}
    \end{column}
\end{columns}
\end{frame}

\begin{frame}[label={sec:visualizationspath}]{Carte des trajets entre deux stations}
\begin{columns}
    \begin{column}{0.45\textwidth}
        \begin{block}{Description}
            \begin{itemize}
                \item Permet de saisir une station de départ et une station d’arrivée.
                \item Trace le chemin le plus court entre ces deux points.
                \item La carte montre le trajet entre les deux stations spécifiées.
                \item Exemple avec "Boutonnet" comme départ et "Garcia Lorca" comme arrivée.
            \end{itemize}
        \end{block}
    \end{column}

    % Deuxième colonne : la carte
    \begin{column}{0.55\textwidth}
        \includegraphics[width=\textwidth]{map_2.png}  % Assurez-vous que l'image est bien dans le même dossier
    \end{column}
\end{columns}

\end{frame}

\begin{frame}[label={sec:visualizationspath}]{Carte informative}
\begin{columns}
    \begin{column}{0.45\textwidth}
        \begin{block}{Description}
            \begin{itemize}
                \item Les stations de vélos sont affichées avec des marqueurs contenant
des informations sur la disponibilité des vélos et des places de stationnement.
            \end{itemize}
        \end{block}
    \end{column}

    % Deuxième colonne : la carte
    \begin{column}{0.55\textwidth}
        \includegraphics[width=\textwidth]{map_info.png}  % Assurez-vous que l'image est bien dans le même dossier
    \end{column}
\end{columns}

\end{frame}

\section{Animation et vidéo}
\label{sec:animation}
\begin{frame}[label={sec:animationframe}]{Animation dynamique des trajets}
\begin{block}{Vidéo des trajets}
    \begin{columns} % Bloc de colonnes
        \begin{column}{0.5\textwidth} % Colonne gauche pour le texte
            \begin{itemize}
                \item \textbf{Vidéo affichant tous les trajets effectués sur une journée.}
                \item \textbf{Date :} 20/09/2024
                \item \textbf{Nombre de trajets affichés :} 75 trajets
            \end{itemize}
        \end{column}
        \begin{column}{0.5\textwidth} % Colonne droite pour l'image
            \begin{center}
                \includegraphics[width=\textwidth]{video.png} % Affiche l'image
            \end{center}
        \end{column}
    \end{columns}
\end{block}
\end{frame}

\section{Défis et perspectives}
\label{sec:challenges}
\begin{frame}[label={sec:challengesframe}]{Défis et perspectives}
\begin{columns}
    \begin{column}{0.45\textwidth} % Colonne pour les défis
        \textbf{Défis rencontrés :}
        \begin{itemize}
            \item Gestion des encodages et des données incomplètes.
            \item Optimisation des temps de calcul pour les visualisations et animations.
        \end{itemize}
    \end{column}
    \begin{column}{0.55\textwidth} % Colonne pour les perspectives
        \textbf{Perspectives :}
        \begin{itemize}
            \item Étendre le projet à d’autres villes ou périodes pour des analyses comparatives.
            \item Intégrer des prédictions en temps réel pour anticiper les flux de trajets.
        \end{itemize}
    \end{column}
\end{columns}
\end{frame}
%%%%%%%%%%%%%%%%%%%%%%%%%%%%%%%%%%%%%%%%%%%%%%%%% A REMPLIR (PARTIE PREDICTION ET ANALYSE DES DONNEES (graphe, croquis etc))

\begin{frame}[label={sec:conclusion}]{Conclusion}
\begin{itemize}
    \item Analyse approfondie des trajets et création d'outils interactifs.
    \item Perspectives prometteuses pour améliorer la gestion des Velomagg.
\end{itemize}
\end{frame}

\end{document}
